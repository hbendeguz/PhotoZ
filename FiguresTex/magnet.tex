%\documentclass{article}
\documentclass{standalone}
\usepackage{tikz}
\usepackage[magyar]{babel}
\begin{document}
\pagestyle{empty}

\def\layersep{2.5cm}

\begin{tikzpicture}[shorten >=0pt,->,draw=black!50, node distance=\layersep]
    \tikzstyle{every pin edge}=[<-,shorten <=1pt]
    \tikzstyle{neuron}=[circle,fill=black!25,minimum size=17pt,inner sep=0pt]
    \tikzstyle{box}=[rectangle,fill=black!25,minimum size=17pt,inner sep=0pt]
    \tikzstyle{input neuron}=[neuron, draw=black!50 , fill=white!50];
    \tikzstyle{im neuron}=[box, draw=black!50 , fill=white!50];
    \tikzstyle{output neuron}=[neuron, draw=black!50 , fill=white!50];
    \tikzstyle{hidden neuron}=[neuron, draw=black!50, fill=white!50];%blue!50];
    \tikzstyle{annot} = [text width=4em, text centered]

    % This is the same as writing \foreach \name / \y in {1/1,2/2,3/3,4/4}
     \foreach \name / \y in {1,...,1} 
        \node[im neuron, pin=left:5 csatorn\'as k\'ep] (I-\name) at (-1,-\y) {};

    % Draw the input layer nodes
    \foreach \name / \y in {1,...,1} 
    % This is the same as writing \foreach \name / \y in {1/1,2/2,3/3,4/4}
        \node[im neuron, pin=left:] (I-\name) at (0,-\y) {};
            
          
    \foreach \name / \y in {2,...,6} 
    % This is the same as writing \foreach \name / \y in {1/1,2/2,3/3,4/4}
        \node[input neuron, pin=left:] (I-\name) at (0,-\y) {};
       \foreach \name / \y in {2,...,2} 
    % This is the same as writing \foreach \name / \y in {1/1,2/2,3/3,4/4}
        \node[input neuron, pin=left:Magnitud\'o u] (I-\name) at (0,-\y) {};
        \foreach \name / \y in {3,...,3} 
    % This is the same as writing \foreach \name / \y in {1/1,2/2,3/3,4/4}
        \node[input neuron, pin=left:Magnitud\'o g] (I-\name) at (0,-\y) {};
\foreach \name / \y in {4,...,4} 
    % This is the same as writing \foreach \name / \y in {1/1,2/2,3/3,4/4}
        \node[input neuron, pin=left:Magnitud\'o r] (I-\name) at (0,-\y) {};
      \foreach \name / \y in {5,...,5} 
    % This is the same as writing \foreach \name / \y in {1/1,2/2,3/3,4/4}
        \node[input neuron, pin=left:Magnitud\'o i] (I-\name) at (0,-\y) {};
         \foreach \name / \y in {6,...,6} 
    % This is the same as writing \foreach \name / \y in {1/1,2/2,3/3,4/4}
        \node[input neuron, pin=left:Magnitud\'o z] (I-\name) at (0,-\y) {};

        

	 
    % Draw the hidden layer nodes
    \foreach \name / \y in {1,...,1}
        \path[yshift=-2cm]
            node[hidden neuron] (H-\name) at (\layersep,-\y cm) {};
	
    % Draw the output layer node
    \node[output neuron,pin={[pin edge={->}]right:V\" or\"oseltol\' od\' as \'ert\'ek}, right of=H-1] (O) {};

    % Connect every node in the input layer with every node in the
    % hidden layer.
    \foreach \source in {1,...,6}
        \foreach \dest in {1,...,1}
            \path (I-\source) edge (H-\dest);

    % Connect every node in the hidden layer with the output layer
    \foreach \source in {1,...,1}
        \path (H-\source) edge (O);

    % Annotate the layers
    \node[annot,above of=H-1, node distance=1cm] (hl) {Teljesen \"oszek\"ot\"ott r\'eteg};
    \node[annot,left of=hl] {};
    \node[annot,right of=hl] {};
\end{tikzpicture}
% End of code
\end{document}